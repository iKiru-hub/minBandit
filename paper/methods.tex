

\section{Methods}

% brief outline of the section
\noindent The following section is organized as follows. First, we introduce a formalization of general problem setting, together with the variants considered in this work. Then, we outline the architecture of the our model and how it can be mapped to neurobiology. Finally, we describe the learning procedure,
and showcase its dynamics in a simple example.

% mathematical formulation of the k-armed bandit problem.
\subsection{Binomial K-armed bandit problem}
\hfill \break
\noindent The standard formulation of the task is structured as a set of $\{1\dots K\}$ levers (or arms), with an associated reward distribution $\mathbf{p}=\{p_{1}, \ldots p_{K}\}$. At each iteration, the agent pulls a lever and collect a possible reward drawn as a Bernoulli variable $R\sim
\mathcal{B}(\{0,1\},p_{k})$. The agent's objective is maximizing the total reward
$\sum^{T}_{t} R_{t}$, after a certain number $T$ of trials.
Importantly, the agent is unaware of the true reward probability distribution, and thus has to make its decisions following a certain policy, denoted as $\omega$. In the reinforcement learning literature, the policy is often defined as
a distribution over the actions, here the levers $K$, given the current state, which in this case can be the history of past actions and rewards up to time $t\leq T$.
Given the inherent stochasticity of the feedbacks from the environment, the definition of the policy is affected by the so-called exploration-exploitation trade-off, which here is phrased as the contrast between the option of the lever with the known highest expected reward versus the option to explore other levers, so to gather more information.
A common approach is the $\epsilon-$greedy policy, where the choice to explore is selected with a probability $\epsilon$.
Moreover, it is often preferable to have a more explorative behaviour early during the training, with the intent to have a good sample size for the empirical reward distribution, which can be later exploited for maximizing reward.

% regret
Another important concept in multi-armed bandit problems is \textit{regret}. Intuitively, it is defined as the deviation of the total reward obtained by the agent from the optimal reward that could have been obtained by always choosing the lever with the highest expected reward.
Formally, the regret is defined as:
\begin{equation}
    \rho=R^{*} - \sum^{T}_{t} R_{t}
\end{equation}

\noindent where $R^{*}$ is the reward obtained by always choosing the lever with the highest expected reward $R^{*}=T\max_{k}\{p_{k}\}$, and $R_{t}$ is the empirical reward obtained up to time $t$ by following policy $\omega$ as $R_{t}=\sum^{T}_{t=1}\omega_{\theta}(t)$.
The regret is a measure of the performance of the agent, and it is often used to compare different algorithms. The goal of the agent is to minimize the regret, and thus maximize the total reward.


\subsection{Model description}
The model is constructed as a rate network of two populations of neurons \textit{M} and \textit{P}, the former representing the memory trace of the \textit{K} available options (\textit{i.e.} the bandits), and the latter representing the value of the options under the current policy.
More formally, the model is described by a set of coupled ordinary differential equations (ODEs) that capture the decision-making process in two distinct neural spaces.
The first equation tracks the evolution of the neural activity $\textbf{u}$ of \textit{M}, while the second tracks the activity $\textbf{v}$ of the \textit{P}. The time constants $\tau$ are the same for both equations and are set to $\tau = 10$ ms.


\begin{equation}
\begin{aligned}
    \tau \dot{\textbf{u}}&= -\textbf{u} + \textbf{v} + \textbf{I}_{\text{ext}} \\
    \tau \dot{\textbf{v}}&= -\textbf{v} + \textbf{z} \odot\textbf{u}
\end{aligned}
\end{equation}

\noindent The external input $\textbf{I}_{\text{ext}}$ is a constant input that is used to set the initial conditions of the neural activity $\textbf{u}$.
The term $\textbf{z}$ is a vector that weights the contribution of the active options $\textbf{u}$ to the value representation $\textbf{v}$, and functionally it is the core of the policy adopted by the model.
In practice, $\textbf{z}$ defined as a function of the synaptic weights $\textbf{W}^{MP}$ from \textit{M} to \textit{P} as $\textbf{z} = \Phi_v(\textbf{W}^{MP})$. Importantly, the connections are not fully connected, but rather are simply one-to-one mapping between the corresponding neurons in each
population, such that the weight matrix $\textbf{W}^{MP}$ is simply a matrix $K\times 1$, namely a vector.
The function $\Phi_v$ is a chosen to be a sum of a generalized sigmoid and a Gaussian, whose contributions are weighted by a parameter $r$:

\begin{equation*}
    \Phi_v(x) = r\gamma_{1} \frac{1}{1 + e^{-\beta(x-\alpha)}} + (1-r)\gamma_{2} \exp\left(-\frac{(x-\mu)^2}{2\sigma^2}\right)
\end{equation*}

\noindent The motivation behind this choice is to express a function that possesses a bounded region (depending on $\mu,\,\sigma$) at a high/low peak (depeding on the value of $\gamma_{2}$), and a continuous transition to a constant value (depending on the steepness of the sigmoid $\beta$, shift
$\alpha$, and intensity $\gamma_{1}$).

\hfill \break
\textbf{Option selection} \\
The decision-making process within a single round is structured in two distinct phases. Initially, the model receives a constant external input targeting all neurons in the memory population \textit{M} equally.
During this phase, $I_{\text{ext}}$ works as an equilibrium value while the reciprocal interactions with population \textit{P} push $\textbf{u}$ to different values, depending on the current policy encoded in $\textbf{z}$. However, in the early rounds the weights $\textbf{W}^{MP}$ are zero, and thus
the contribution from \textit{P} is null. After a fixed amount of time $\sim 5 \text{s}$, the second phase begins. Here, the external input is removed and the model is left to evolve autonomously, and since there are no recurrent connections in neither population the dynamics is entirely driven by their coupling. \\
A selection $\hat{k}$ is sampled after another fixed amount of time $\sim 5 \text{s}$, and it is defined according to the following rule:

\begin{equation*}
    \hat{k} =
    \left\{
        \begin{array}{ll}
            \text{argmax}_{k}\{\textbf{v}\} & \text{\textit{if}}\; \text{argmax}_{k} \{\textbf{v}\} = \text{argmax}_{k} \{\textbf{u}\} \\
            \text{random}(K) & \text{\textit{otherwise}}
        \end{array}
    \right.
\end{equation*}

\noindent The selection rule is simple: if the value representation $\textbf{v}$ is in agreement with the memory trace $\textbf{u}$, then the option with the highest value is selected. Otherwise, a random option is chosen. This rule is a way to express the exploration-exploitation trade-off, and it is dependent on the current policy $\textbf{z}$.

\hfill \break
\textbf{Learning} \\
Given a selected option, the environment (bandit) samples and returns a reward $R\in [0, 1]$.
Then, the connections $\textbf{W}^{MP}$ for the neuron corresponding to the option $k$ are updated according to the following plasticity rule:

\begin{equation}
    \Delta \textbf{W}^{MP}_{k} = \tilde{\eta}_{k} \left(R\cdot W^{+}- \textbf{W}^{MP}_{k}\right)
\end{equation}

\noindent
Where $W^{+}$ is a constant value that sets the upper bound for the synaptic weights, and it is set to $W^{+} = 5$, while $\tilde{\eta}_{k}$ is the learning rate for the option $k$ determined by a function of the current weights $\textbf{W}^{MP}_{k}$ and its shape is the same as $\Phi_{v}$, but with
different parameters.



